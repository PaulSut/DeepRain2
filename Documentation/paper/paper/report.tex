\documentclass[oneside]{htwg-report}

\include{header/header}

\addbibresource{./bib/report.bib}

\begin{document}

\pagenumbering{gobble}

%% 'reporttype' add background elements to the cover / front page
%% possible values are:
%% bachelor	--> B S C
%% master	--> M S C
%% other		--> none
\reporttype{master}

\reporttypetext{Teamproject (Master 3. semester)}

\newcommand{\verfasserA}{Simon Christofzik}
\newcommand{\verfasserB}{Paul Sutter}
\newcommand{\verfasserC}{Till Reitlinger}
\newcommand{\thema}{DeepRain: Rain forecast with neural networks and the visualization of these in an App}
\newcommand{\hoschschule}{HTWG Konstanz - University of Applied Sciences}
\newcommand{\institut}{HTWG Konstanz - Institute for Optical Systems}
\newcommand{\prueferA}{Prof. Dr. Oliver Dürr}


\title[Teamprojektthema]{\thema}

\doclocation{Konstanz}
\docdate{10. September 2020}

\makecover[]

\chapter*{Extended Abstract}

\begin{center}
	\begingroup
	\renewcommand*{\arraystretch}{1}
	\rowcolors{2}{white}{white}
	{\makeatletter	
		\begin{tabular}{p{3.2cm}p{9.6cm}}
			Topic: & \thema \\
			& \\
			Team members: & \verfasserA, \verfasserB, \verfasserC \\
			& \\
			Advisor: & \hoschschule \newline \institut \newline \prueferA \\
			& \\
		\end{tabular}
		
		\makeatother}
	\endgroup
\end{center}

\bigskip

\noindent

The goal of the present work is to examine whether it is possible to calculate a rainfall forecast with limited resources and to make it available to users. 
For the calculation of the rain forecast neural networks were used. 
The required historical and current radar data were obtained from the German Weather Service and then analyzed and processed. 
Furthermore, an app was developed in which the rain forecasts are visualized.  
It also offers the possibility to notify the user in case of imminent rain. 
All the code and the full length documentation can be found on GitHub: \url{https://github.com/PaulIVI/DeepRain2}.




\twocolumn
\section*{Introduction}
    \begin{sloppypar}
        \tolerance 9999
        Even today, rain forecasts are still very computationally complex and relatively inaccurate. 
        Therefore, it may make sense to make such predictions with the help of neural networks. 
        These do not require as much computing power and can recognize a pattern in the often chaotic data even without complex physical models.
    \end{sloppypar}

\section*{Data}

\section*{Data preprocessing}

\begin{figure}[ht]
\centering
\includegraphics[width=0.8\linewidth]{../pics/UNet_Biomedical}
\caption{The image is taken from the university of Freiburg~\cite{ronneberger2015u}}
\end{figure}

\begin{sloppypar}
\tolerance 9999
\noindent
\end{sloppypar}

\section*{DeepRain Application}
    \begin{sloppypar}
        The DeepRain App was developed with Flutter. 
        Firebase is used for the user administration, the forecast image transmission and the sending of rain warnings (push notifications). 
        The App basically consists of three screens. One screen is the settings. 
        In the settings you can change the region, configure the rain warning function and some other settings.
        On the Rain Map screen the current and future rain situation is displayed using the forecast PNGs. With a slider you can display the rain situation at different times.
        On the Prediction List screen the current and future rain situation is displayed quantitatively. The rain values are read from the respective pixel of the region in the forecast image.   
    \end{sloppypar}
\subsection*{Calculate rain intense}
    \begin{sloppypar}
        To display the rain intensity of a certain region in a list, the color value of the pixel belonging to the region must be read out. 
        To calculate the pixel of the region in the forecast image, an algorithm was developed, which finds the corresponding pixel in the forecast image with the help of a latitude and longitude coordinate. 
        This is only possible because the latitude and longitude coordinates of each pixel are known. 
        The algorithm starts at any pixel, checks the current distance to the target coordinates and the distance of the neighboring pixels to the target coordinates. 
        If an neighboring pixel has a smaller distance, the pixel is updated. If no more improvement can be achieved, the correct pixel is found. 
    \end{sloppypar}

\section*{Pipeline}

\begin{figure}[ht]
    \centering
    \includegraphics[width=0.8\linewidth]{../pics/UNet_Biomedical}
    \caption{The image is taken from the university of Freiburg~\cite{ronneberger2015u}}
\end{figure}



\begin{figure}[ht]
    \centering
    \includegraphics[width=0.8\linewidth]{../pics/UNet_Biomedical}
    \caption{The image is taken from the university of Freiburg~\cite{ronneberger2015u}}
\end{figure}

\section*{Conclusion and Future Work}


\end{document}

