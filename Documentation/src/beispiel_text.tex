% einzelne Kapitel werden hier eingebunden
\input{1_einleitung}
\newpage

\input{2_kap1}
\newpage

\input{3_kap2}
\newpage

% hier können weitere Kapitel angelegt und eingetragen werden
% ....

\input{7_ausblick}
\newpage

\input{8_fazit}

\input{beispiel}

% einfacher Zeilenabstand
\singlespacing
% Literaturliste soll im Inhaltsverzeichnis auftauchen
\newpage
\addcontentsline{toc}{section}{Literaturverzeichnis}
% Literaturverzeichnis anzeigen
\renewcommand\refname{Literaturverzeichnis}
\bibliography{Hauptdatei}

%% Index soll Stichwortverzeichnis heissen
% \newpage
% % Stichwortverzeichnis soll im Inhaltsverzeichnis auftauchen
% \addcontentsline{toc}{section}{Stichwortverzeichnis}
% \renewcommand{\indexname}{Stichwortverzeichnis}
% % Stichwortverzeichnis endgültig anzeigen
% \printindex

\onehalfspacing
% evtl. Anhang
\newpage
\addcontentsline{toc}{section}{Anhang}
\fancyhead[L]{Anhang} %Kopfzeile links
\input{anhang/anhang}

% Eidesstattliche Erklärung
\newpage
\addcontentsline{toc}{section}{Eidesstattliche Erklärung}
\input{erklaerung}
